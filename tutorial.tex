%\newpage
%============================================================================================================================

% \section{Ссылки} \label{sect1_2}
% Сошлёмся на библиографию. Одна ссылка: \cite[с.~54]{Sokolov}\cite[с.~36]{Gaidaenko}. Две ссылки: \cite{Sokolov,Gaidaenko}. Много ссылок:  \cite[с.~54]{Lermontov,Management,Borozda} \cite{Lermontov,Management,Borozda,Marketing,Constitution,FamilyCode,Gost.7.0.53,Razumovski,Lagkueva,Pokrovski,Sirotko,Lukina,Methodology,Encyclopedia,Nasirova,Berestova,Kriger}. И ещё немного ссылок: \cite{Article,Book,Booklet,Conference,Inbook,Incollection,Manual,Mastersthesis,Misc,Phdthesis,Proceedings,Techreport,Unpublished}. \cite{medvedev2006jelektronnye, CEAT:CEAT581, doi:10.1080/01932691.2010.513279,Gosele1999161,Li2007StressAnalysis, Shoji199895,test:eisner-sample,AB_patent_Pomerantz_1968,iofis_patent1960}

%Попытка реализовать несколько ссылок на конкретные страницы для стандартной реализации:[\citenum{Sokolov}, с.~54; \citenum{Gaidaenko}, с.~36].

%Несколько источников мультицитата \cites[vii--x, 5, 7]{Sokolov}[v--x, 25, 526]{Gaidaenko} поехали дальше

% Ссылки на собственные работы:~\cite{vakbib1, confbib1}
% 
% Сошлёмся на приложения: Приложение \ref{AppendixA}, Приложение \ref{AppendixB2}.
% 
% Сошлёмся на формулу: формула \eqref{eq:equation1}.
% 
% Сошлёмся на изображение: рисунок \ref{img:knuth}.

%\newpage
%============================================================================================================================

% \section{Формулы} \label{sect1_3}
% 
% Благодаря пакету \textit{icomma}, \LaTeX~одинаково хорошо воспринимает в качестве десятичного разделителя и запятую ($3,1415$), и точку ($3.1415$).
% 
% \subsection{Ненумерованные одиночные формулы} \label{subsect1_3_1}
% 
% Вот так может выглядеть формула, которую необходимо вставить в строку по тексту: $x \approx \sin x$ при $x \to 0$.
% 
% А вот так выглядит ненумерованая отдельностоящая формула c подстрочными и надстрочными индексами:
% \[
% (x_1+x_2)^2 = x_1^2 + 2 x_1 x_2 + x_2^2
% \]
% 
% При использовании дробей формулы могут получаться очень высокие:
% \[
%   \frac{1}{\sqrt{2}+
%   \displaystyle\frac{1}{\sqrt{2}+
%   \displaystyle\frac{1}{\sqrt{2}+\cdots}}}
% \]
% 
% В формулах можно использовать греческие буквы:
% \[
% \alpha\beta\gamma\delta\epsilon\varepsilon\zeta\eta\theta\vartheta\iota\kappa\lambda\\mu\nu\xi\pi\varpi\rho\varrho\sigma\varsigma\tau\upsilon\phi\varphi\chi\psi\omega\Gamma\Delta\Theta\Lambda\Xi\Pi\Sigma\Upsilon\Phi\Psi\Omega
% \]
% 
% %\newpage
% %============================================================================================================================
% 
% \subsection{Ненумерованные многострочные формулы} \label{subsect1_3_2}
% 
% Вот так можно написать две формулы, не нумеруя их, чтобы знаки равно были строго друг под другом:
% \begin{align}
%   f_W & =  \min \left( 1, \max \left( 0, \frac{W_{soil} / W_{max}}{W_{crit}} \right)  \right), \nonumber \\
%   f_T & =  \min \left( 1, \max \left( 0, \frac{T_s / T_{melt}}{T_{crit}} \right)  \right), \nonumber
% \end{align}
% 
% Выровнять систему ещё и по переменной $ x $ можно, используя окружение \verb|alignedat| из пакета \verb|amsmath|. Вот так: 
% \[
%     |x| = \left\{
%     \begin{alignedat}{2}
%         &&x, \quad &\text{eсли } x\geqslant 0 \\
%         &-&x, \quad & \text{eсли } x<0
%     \end{alignedat}
%     \right.
% \]
% Здесь первый амперсанд  означает выравнивание по~левому краю, второй "--- по~$ x $, а~третий "--- по~слову <<если>>. Команда \verb|\quad| делает большой горизонтальный пробел. 
% 
% Ещё вариант:
% \[
%     |x|=
%     \begin{cases}
%     \phantom{-}x, \text{если } x \geqslant 0 \\
%     -x, \text{если } x<0
%     \end{cases}
% \]
% 
% Можно использовать разные математические алфавиты:
% \begin{align}
% \mathcal{ABCDEFGHIJKLMNOPQRSTUVWXYZ} \nonumber \\
% \mathfrak{ABCDEFGHIJKLMNOPQRSTUVWXYZ} \nonumber \\
% \mathbb{ABCDEFGHIJKLMNOPQRSTUVWXYZ} \nonumber
% \end{align}
% 
% Посмотрим на систему уравнений на примере аттрактора Лоренца:
% 
% \[ 
% \left\{
%   \begin{array}{rl}
%     \dot x = & \sigma (y-x) \\
%     \dot y = & x (r - z) - y \\
%     \dot z = & xy - bz
%   \end{array}
% \right.
% \]
% 
% А для вёрстки матриц удобно использовать многоточия:
% \[ 
% \left(
%   \begin{array}{ccc}
%   	a_{11} & \ldots & a_{1n} \\
%   	\vdots & \ddots & \vdots \\
%   	a_{n1} & \ldots & a_{nn} \\
%   \end{array}
% \right)
% \]


%\newpage
%============================================================================================================================
% \subsection{Нумерованные формулы} \label{subsect1_3_3}
% 
% А вот так пишется нумерованая формула:
% \begin{equation}
%   \label{eq:equation1}
%   e = \lim_{n \to \infty} \left( 1+\frac{1}{n} \right) ^n
% \end{equation}
% 
% Нумерованых формул может быть несколько:
% \begin{equation}
%   \label{eq:equation2}
%   \lim_{n \to \infty} \sum_{k=1}^n \frac{1}{k^2} = \frac{\pi^2}{6}
% \end{equation}
% 
% Впоследствии на формулы (\ref{eq:equation1}) и (\ref{eq:equation2}) можно ссылаться.
% 
% Сделать так, чтобы номер формулы стоял напротив средней строки, можно, используя окружение \verb|multlined| (пакет \verb|mathtools|) вместо \verb|multline| внутри окружения \verb|equation|. Вот так:
% \begin{equation} % \tag{S} % tag - вписывает свой текст 
%   \label{eq:equation3}
%     \begin{multlined}
%         1+ 2+3+4+5+6+7+\dots + \\ 
%         + 50+51+52+53+54+55+56+57 + \dots + \\ 
%         + 96+97+98+99+100=5050 
%     \end{multlined}
% \end{equation}

% Используя команду \verb|\labelcref| из пакета \verb|cleveref|, можно
% красиво ссылаться сразу на несколько формул
% (\labelcref{eq:equation1,eq:equation3,eq:equation2}), даже перепутав
% порядок ссылок \verb|(\labelcref{eq:equation1,eq:equation3,eq:equation2})|.

% А это две картинки под общим номером и названием:
% \begin{figure}[ht]
%   \begin{minipage}[ht]{0.49\linewidth}
%     \center{\includegraphics[width=0.5\linewidth]{knuth1} \\ а)}
%   \end{minipage}
%   \hfill
%   \begin{minipage}[ht]{0.49\linewidth}
%     \center{\includegraphics[width=0.5\linewidth]{knuth2} \\ б)}
%   \end{minipage}
%   \caption{Очень длинная подпись к изображению, на котором представлены две фотографии Дональда Кнута}
%   \label{img:knuth}  
% \end{figure}
% 
% Те~же~две картинки под~общим номером и~названием, но с автоматизированной нумерацей подрисунков посредством пакета \verb|subcaption|:
% \begin{figure}[ht]
%     \center{
%         \hfill
%         \subcaptionbox[List-of-Figures entry]{Первый подрисунок\label{img:knuth_2_1}} {\includegraphics[width=0.25\linewidth]{knuth1}}%
%         \hfill       
%         \subcaptionbox{Второй подрисунок\label{img:knuth_2_2}} {\includegraphics[width=0.25\linewidth]{knuth2}}
%         \hfill
%     }
%     
%     \onehalfspacing{% внутри окружения figure набитый текст идёт через одинарный 
%интервал, потому применяем эту команду пакета setspace. Возможно это временный "костыль", 
%до появления соответствующей настройки в преамбуле
%     Подрисуночный текст, описывающий обозначения, например. Согласно ГОСТ 2.105, 
%пункт 4.3.1, располагается перед наименованием рисунка.
%     }
%     \caption{Очень длинная подпись к второму изображению, на котором представлены две фотографии Дональда Кнута} % Этот текст попадает в названия рисунков в списке рисунков
%     \label{img:knuth_2}  
% \end{figure}
% 
% 
% На рисунке~\ref{img:knuth_2_1} показан Дональд Кнут без головного убора. 
%На рисунке~\ref{img:knuth_2}\subref*{img:knuth_2_2}  показан Дональд Кнут в головном уборе.

%\newpage
%============================================================================================================================
%\section{Пример вёрстки списков} \label{sect2_3}

% \noindent Нумерованный список:
% \begin{enumerate}
%   \item Первый пункт.
%   \item Второй пункт.
%   \item Третий пункт.
% \end{enumerate}
% 
% \noindent Маркированный список:
% \begin{itemize}
%   \item Первый пункт.
%   \item Второй пункт.
%   \item Третий пункт.
% \end{itemize}
% 
% \noindent Вложенные списки:
% \begin{itemize}
%   \item Имеется маркированный список.
%   \begin{enumerate}
%     \item В нём лежит нумерованный список,
%     \item в котором
%     \begin{itemize}
%       \item лежит ещё один маркированный список.
%     \end{itemize}    
%   \end{enumerate}
% \end{itemize}


% \section{Пробелы}
% 
% В~русском наборе принято:
% \begin{itemize}
%     \item единицы измерения, знак процента отделять пробелами от~числа: 10~кВт, 15~\% (согласно ГОСТ 8.417, раздел 8);
%     \item $\tg 20^\circ$, но: 20~${}^\circ$C (согласно ГОСТ 8.417, раздел 8);
%     \item знак номера, параграфа отделять от~числа: №~5, \S~8;
%     \item стандартные сокращения: т.\:е., и~т.\:д., и~т.\:п.;
%     \item неразрывные пробелы в~предложениях.
% \end{itemize}

% \section{Математика}
% 
% Русская традиция начертания греческих букв отличается от~западной. Это исправляется серией \verb|\renewcommand|.
% \begin{itemize}
%     \item[До:] $ \epsilon \ge \phi$, $\phi \leq \epsilon$, $\kappa \in \emptyset$.
%     \renewcommand{\epsilon}{\ensuremath{\varepsilon}}
%     \renewcommand{\phi}{\ensuremath{\varphi}}
%     \renewcommand{\kappa}{\ensuremath{\varkappa}}
%     \renewcommand{\le}{\ensuremath{\leqslant}}
%     \renewcommand{\leq}{\ensuremath{\leqslant}}
%     \renewcommand{\ge}{\ensuremath{\geqslant}}
%     \renewcommand{\geq}{\ensuremath{\geqslant}}
%     \renewcommand{\emptyset}{\varnothing}
%     \item[После:] $\epsilon \ge \phi$, $\phi \leq \epsilon$, $\kappa \in \emptyset$.
% \end{itemize}
% 
% Кроме того, принято набирать греческие буквы вертикальными, что решается подключением пакета \verb|upgreek| (см. закомментированный блок в \verb|userpackages.tex|) и~аналогичным переопределением в преамбуле (см. закомментированный блок в \verb|userstyles.tex|).
% 
% 
% \section{Кавычки}
% В английском языке приняты одинарные и двойные кавычки в~виде ‘...’ и~“...”. В России приняты французские («...») и~немецкие („...“) кавычки (они называются «ёлочки» и~«лапки», соответственно). <<Лапки>> обычно используются внутри ,,ёлочек``, например, <<... наш гордый ,,Варяг``...>>.
% 
% Французкие левые и правые кавычки набираются
% как лигатуры \verb|<<| и \verb|>>|, а~немецкие левые и правые кавычки набираются как лигатуры \verb|,,| и \verb|‘‘| (\verb|``|).
% 
% Вместо лигатур или команд с~активным символом "\ можно использовать команды \verb|\glqq| и \verb|\grqq| для набора немецких кавычек и команды \verb|\flqq| и \verb|\frqq| для набора французских кавычек. Они определены в пакете \verb|babel|.
% 
% \section{Тире}
% %  babel+pdflatex по умолчанию, в polyglossia надо включать опцией (и перекомпилировать с удалением временных файлов)
% Команда \verb|"---| используется для печати тире в тексте. Оно несколько короче английского длинного тире. Кроме того, команда задаёт небольшую жёсткую отбивку от слова, стоящего перед тире. При этом, само тире не отрывается от слова. После тире следует такая же отбивка от текста, как и перед тире. При наборе текста между словом и командой, за которым она следует, должен стоять пробел.
% 
% В составных словах, таких, как <<Закон Менделеева"--~Клапейрона>>, для печати тире надо 
%использовать команду \verb|"--~|. Она ставит более короткое, по~сравнению с~английским, 
%тире и позволяет делать переносы во втором слове. При~наборе текста команда \verb|"--~| не 
%отделяется пробелом от слова, за которым она следует (\verb|Менделеева"--~|). Следующее за 
%командой слово может быть  отделено от~неё пробелом или перенесено на другую строку.
% 
% Если прямая речь начинается с~абзаца, то перед началом её печатается тире командой
% \verb|"--*|. Она печатает русское тире и жёсткую отбивку нужной величины перед текстом.
% 
% \section{Дефисы и переносы слов}
% %  babel+pdflatex по умолчанию, в polyglossia надо включать опцией (и перекомпилировать с 
%удалением временных файлов)
% Для печати дефиса в~составных словах введены две команды. Команда~\verb|"~| печатает дефис и~запрещает делать переносы в~самих словах, а~команда \verb|"=| печатает дефис, оставляя \TeX ’у право делать переносы в~самих словах.
% 
% В отличие от команды \verb|\-|, команда \verb|"-| задаёт место в~слове, где можно делать перенос, не~запрещая переносы и~в~других местах слова.
% 
% Команда \verb|""| задаёт место в~слове, где можно делать перенос, причём дефис при~переносе в~этом месте не~ставится.
% 
% Команда \verb|",| вставляет небольшой пробел после инициалов с~правом переноса в~фамилии.
% 
% \section{Текст из панграмм и формул}
% 
% \begin{multline*}
% \mathsf{Pr}(\digamma(\tau))\propto\sum_{i=4}^{12}\left( \prod_{j=1}^i\left( \int_0^5\digamma(\tau)e^{-\digamma(\tau)t_j}dt_j \right)\prod_{k=i+1}^{12}\left( \int_5^\infty\digamma(\tau)e^{-\digamma(\tau)t_k}dt_k\right)C_{12}^i \right)\propto\\
% \propto\sum_{i=4}^{12}\left( -e^{-1/2}+1\right)^i\left( e^{-1/2}\right)^{12-i}C_{12}^i \approx 0.7605,\quad \forall\tau\neq\overline{\tau}
% \end{multline*}
% Квуй ыёюз омниюм йн. Экз алёквюам кончюлату квуй, ты альяквюам ёнвидюнт пэр. Зыд нэ коммодо пробатуж. Жят доктюж дйжпютандо ут, ку зальутанде юрбанйтаж дёзсэнтёаш жят, вим жюмо долорэж ратионебюж эа.
% 
% Ад ентэгры корпора жплэндидэ хаж. Эжт ат факэтэ дычэрунт пэржыкюти. Нэ нам доминг пэрчёус. Ку квюо ёужто эррэм зючкёпит. Про хабэо альбюкиюс нэ.
% \[
% \begin{pmatrix}
% a_{11} & a_{12} & a_{13} \\
% a_{21} & a_{22} & a_{23}
% \end{pmatrix}
% \]
% 
% \[
% \begin{vmatrix}
% a_{11} & a_{12} & a_{13} \\
% a_{21} & a_{22} & a_{23}
% \end{vmatrix}
% \]
% 
% \[
% \begin{bmatrix}
% a_{11} & a_{12} & a_{13} \\
% a_{21} & a_{22} & a_{23}
% \end{bmatrix}
% \]
% Про эа граэки квюаыквуэ дйжпютандо. Ыт вэл тебиквюэ дэфянятйоныс, нам жолюм квюандо мандамюч эа. Эож пауло лаудым инкедыринт нэ, пэрпэтюа форынчйбюж пэр эю. Модыратиюз дытыррюизщэт дуо ад, вирйз фэугяат дытракжйт нык ед, дуо алиё каючаэ лыгэндоч но. Эа мольлиз юрбанйтаж зигнёфэрумквюы эжт.
% 
% Про мандамюч кончэтытюр ед. Трётанё прёнкипыз зигнёфэрумквюы вяш ан. Ат хёз эквюедым щуавятатэ. Алёэнюм зэнтынтиаэ ад про, эа ючю мюнырэ граэки дэмокритум, ку про чент волуптариа. Ыльит дыкоры аляквюид еюж ыт. Ку рыбюм мюндй ютенам дуо.
% \begin{align*}
% 2\times 2 &= 4 & 6\times 8 &= 48 \\
% 3\times 3 &= 9 & a+b &= c\\
% 10 \times 65464 &= 654640 & 3/2&=1,5
% \end{align*}
% 
% \begin{equation}
% \begin{aligned}
% 2\times 2 &= 4 & 6\times 8 &= 48 \\
% 3\times 3 &= 9 & a+b &= c\\
% 10 \times 65464 &= 654640 & 3/2&=1,5
% \end{aligned}
% \end{equation}
% 
% 
% %Большая фигурная скобка только справа
% \[\left.                                                          %ВАЖНО: точка после слова left делает скобку неотображаемой
% \begin{aligned}
% 2 \times x &= 4 \\
% 3 \times y &= 9 \\
% 10 \times 65464 &= z
% \end{aligned}\right\} \]
% 
% 
% \[ \frac{m_{t\vphantom{y}}^2}{L_t^2} = \frac{m_{x\vphantom{y}}^2}{L_x^2} + \frac{m_y^2}{L_y^2} + \frac{m_{z\vphantom{y}}^2}{L_z^2} \]
% 
