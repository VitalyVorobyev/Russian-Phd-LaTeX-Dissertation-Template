\chapter*{Введение}							% Заголовок
\addcontentsline{toc}{chapter}{Введение}	% Добавляем его в оглавление

\newcommand{\actuality}{}
\newcommand{\progress}{}
\newcommand{\aim}{\textbf{Целью}}
\newcommand{\tasks}{\textbf{задачи}}
\newcommand{\defpositions}{\textbf{Основные положения, выносимые на~защиту:}}
\newcommand{\novelty}{\textbf{Научная новизна:}}
\newcommand{\influence}{\textbf{Научная и практическая значимость}}
\newcommand{\reliability}{\textbf{Степень достоверности}}
\newcommand{\probation}{\textbf{Апробация работы.}}
\newcommand{\contribution}{\textbf{Личный вклад.}}
\newcommand{\publications}{\textbf{Публикации.}}

{\actuality} Эксперименты \belle и \babar, начавшие в $1999$ году набирать данные на $B$-фабриках \kekb и \pepii, соответственно, существенно продвинули понимание физики тяжелых кварков.  Ключевым результатом работы этих экспериментов стало наблюдение и детальное изучение нарушения \cpconj-симметрии в распадах $B$-мезонов.  Все обнаруженные \cpconj-нарушающие явления находятся в согласии с механизмом \cpconj-нарушения Кобаяши-Маскавы (\km) для слабых заряженных токов.

Нарушение \cpconj-симметрии важно не только с точки зрения описания взаимодействий элементарных частиц. Нарушение этой симметрии необходимо для описания бариогенеза и механизма формирования преобладания материи над антиматерией в видимой части Вселенной.  Есть основания полагать, что требуемая для описания бариогенеза степень нарушения \cpconj-симметрии не может быть обеспечена \km-механизмом, поэтому прецизионное измерение параметров этого механизма и поиск новых механизмов нарушения \cpconj-симметрии являются актуальными задачами экспериментальной физики высоких энергий.

Анализ многочастичных распадов является замечательным инструментом для измерения параметров \cpconj-нарушения и осцилляций нейтральных мезонов.  В некоторых случаях использование многочастичных распадов является необходимым условием для измерения величины параметра (а не установления факта отличия его величины от нуля).  Особенностью таких измерений является необходимость обладать информацией о не наблюдаемой непосредственно фазе амплитуды многочастичного распада.  Амплитуда распада не может быть получена из первых принципов из-за непертурбативных эффектов квантовой хромодинамики.  Эта проблема может быть решена с помощью построения феноменологической модели амплитуды распада и вычисления фазы с помощью этой модели.  
Такой подход, однако, неизбежно приводит к неустранимой и плохо контролируемой модельной неопределенности, которая может стать определяющей при выполнении прецизионных измерений в экспериментах~\lhcb и~\belleii.

Альтернативный подход, в котором среднее значение разности фаз амплитуд распадов \dn- и \dnbar-мезонов для определенной области фазового пространства извлекаются из эксперимента, не требует построения модели.  Этот подход может применяться в экспериментах \lhcb, \belleii, а также на Чарм-Тау-фабрике.

\aim\ данной работы является разработка и доказательство практической реализуемости модельно-независимого подхода к измерению параметров смешивания мезонов и параметров нарушения \cpconj-симметрии с использованием многочастичных распадов $D$- и $B$-мезонов.

Для~достижения поставленной цели необходимо было решить следующие {\tasks}:
\begin{enumerate}
  \item Исследовать влияние осцилляций и прямого нарушения \cpconj-симметрии в распадах $D$-мезонов на измеряемую величину \cpconj-нарушающего параметра \gphi модельно-независимо измеряемую в распадах \bdk, \dkpp.
  \item Разработать модельно-независимый метод получения параметров осцилляций и параметров нарушения \cpconj-симметрии в осцилляциях $D$-мезонов.
  \item Разработать модельно-независимый метод получения параметра \cpconj-нарушения \pphi в распадах \bdh, \dbkpp, $h^0\in\{\pin,\eta^{(\prime)},\omega\}$.
  \item Выполнить модельно-независимое измерение параметра \pphi в вышеупомянутом распаде, используя разработанный метод.
\end{enumerate}

\defpositions
%%% Согласно ГОСТ Р 7.0.11-2011:
%% 5.3.3 В заключении диссертации излагают итоги выполненного исследования, рекомендации, перспективы дальнейшей разработки темы.
%% 9.2.3 В заключении автореферата диссертации излагают итоги данного исследования, рекомендации и перспективы дальнейшей разработки темы.
\begin{enumerate}
  \item Изучено влияние осцилляций нейтральных $D$-мезонов на наблюдаемую величину параметра \gphi в модельно-независимом измерении в распадах \bdk, \dkpp и предложена процедура, при которой осцилляции $D$-мезонов смещают наблюдаемую величину не более, чем на~$0.2\grad$.
  \item Показано, что в предположении сохранения \cpconj-симметрии в распадах \dnkpp и при существующих экспериментальных ограничениях на величину этого нарушения, смещение наблюдаемой величины \gphi при модельно-независимом измерении в распадах \bdk, \dkpp, не превосходит $3\grad$.
  \item Показано, что модельно-независимое получение параметра \gphi в распадах \bdk, \dkpp возможно без предположения сохранения \cpconj-симметрии в распадах \dnkpp; при этом не наблюдается существенного снижения статистической чувствительности.
  \item Предложен метод модельно-независимого измерения параметров смешивания и \cpconj-нарушения в смешивании нейтральных $D$-мезонов в процессе $\ep\to \ddbar{}^{*}$ без измерения времени распада~$D$.
  \item Предложен метод модельно-независимого получения параметров смешивания и \cpconj-нарушения в смешивании нейтральных $D$-мезонов в процессе \dstpdpip, \dnkpp с измерением времени распада~$D$.
  \item Предложен метод модельно-независимого измерения \cpconj-нарушающей фазы \pphi в распадах \bdsth, \dbkpp; данный метод позволяет разрешить неопределенность, присущую измерению $2\pphi$ в переходах~\btoccs.
  \item Впервые выполнено модельно-независимое измерение фазы \pphi в распадах \bdsth, \dbkpp и получен результат $\pphi = 11.7\grad\pm7.8\grad\pm2.1\grad$, позволяющий разрешить неопределенность значения $2\pphi$ на уровне достоверности, превышающем $5$ стандартных отклонений.
  \item Подготовлен алгоритм для автоматического измерения характеристик модуля усилителя-формирователя калориметра \belleii, который был использован для проверки характеристик всех изготовленных модулей.
\end{enumerate}
% \begin{enumerate}
%   \item На основе анализа \ldots
%   \item Численные исследования показали, что \ldots
%   \item Математическое моделирование показало \ldots
%   \item Для выполнения поставленных задач был создан \ldots
% \end{enumerate}

\begin{enumerate}
  \item Изучено влияние осцилляций нейтральных $D$-мезонов на наблюдаемую величину параметра \gphi в модельно-независимом измерении в распадах \bdk, \dkpp и предложена процедура, при которой осцилляции $D$-мезонов смещают наблюдаемую величину не более, чем на~$0.2\grad$.
  \item Показано, что в предположении сохранения \cpconj-симметрии в распадах \dnkpp и при существующих экспериментальных ограничениях на величину этого нарушения, смещение наблюдаемой величины \gphi при модельно-независимом измерении в распадах \bdk, \dkpp, не превосходит $3\grad$.
  \item Показано, что модельно-независимое получение параметра \gphi в распадах \bdk, \dkpp возможно без предположения сохранения \cpconj-симметрии в распадах \dnkpp; при этом не наблюдается существенного снижения статистической чувствительности.
  \item Предложен метод модельно-независимого измерения параметров смешивания и \cpconj-нарушения в смешивании нейтральных $D$-мезонов в процессе $\ep\to \ddbar{}^{*}$ без измерения времени распада~$D$.
  \item Предложен метод модельно-независимого получения параметров смешивания и \cpconj-нарушения в смешивании нейтральных $D$-мезонов в процессе \dstpdpip, \dnkpp с измерением времени распада~$D$.
  \item Предложен метод модельно-независимого измерения \cpconj-нарушающей фазы \pphi в распадах \bdsth, \dbkpp; данный метод позволяет разрешить неопределенность, присущую измерению $2\pphi$ в переходах~\btoccs.
  \item Впервые выполнено модельно-независимое измерение фазы \pphi в распадах \bdsth, \dbkpp и получен результат $\pphi = 11.7\grad\pm7.8\grad\pm2.1\grad$, позволяющий разрешить неопределенность значения $2\pphi$ на уровне достоверности, превышающем $5$ стандартных отклонений.
  \item Подготовлен алгоритм для автоматического измерения характеристик модуля усилителя-формирователя калориметра \belleii, который был использован для проверки характеристик всех изготовленных модулей.
\end{enumerate}

\novelty\ впервые выполнено модельно-независимое измерение параметра \pphi в распадах \bdsth, \dbkpp; впервые предложены свободные от модельной неопределенности методы измерения параметров смешивания $D$-мезонов и \cpconj-нарушающего параметра \pphi с использованием многочастичных распадов нейтральных $D$-мезонов.

\influence: предложенный метод измерения параметра \pphi, а также результаты исследования процедуры модельно-независимого измерения параметра \gphi используются и будут использоваться при выполнении прецизионных измерений в экспериментах \babar, \belle, \belleii и \lhcb.  Предложенный метод измерения параметров осцилляций $D$-мезонов может быть использован при выполнении измерений в эксперименте \besiii и в будущих экспериментах на Чарм-Тау-фабрике.

\reliability\ полученных результатов обеспечивается публикацией основных результатов в рецензируемых журналах с высокой цитируемостью. Результаты измерения параметра \pphi находятся в согласии с предыдущим измерением в эксперименте \belle, а также с результатом измерения, выполненного группой \babar.

\probation\ Основные результаты работы докладывались~на научных семинарах в ИЯФ СО РАН и KEK (Цукуба, Япония).  Результаты измерения параметра \pphi были доложены на конференциях XIII Heavy Quarks and Leptons conference (HQL 2016) и 38th International Conference On High Energy Physics (ICHEP 2016).

\contribution\ Изложенные в работе результаты получены автором лично либо при его определяющем вкладе.

\ifthenelse{\equal{\thebibliosel}{0}}{% Встроенная реализация с загрузкой файла через движок bibtex8
    \publications\ Основные результаты по теме диссертации изложены в XX печатных изданиях, 
    X из которых изданы в журналах, рекомендованных ВАК, 
    X "--- в тезисах докладов.%
}{% Реализация пакетом biblatex через движок biber
%Сделана отдельная секция, чтобы не отображались в списке цитированных материалов
    \begin{refsection}%
        \printbibliography[heading=countauthornotvak, env=countauthornotvak, keyword=biblioauthornotvak, section=1]%
        \printbibliography[heading=countauthorvak, env=countauthorvak, keyword=biblioauthorvak, section=1]%
        \printbibliography[heading=countauthorconf, env=countauthorconf, keyword=biblioauthorconf, section=1]%
        \printbibliography[heading=countauthor, env=countauthor, keyword=biblioauthor, section=1]%
        \publications\ Основные результаты по теме диссертации изложены в \arabic{citeauthor} печатных изданиях\nocite{mixing,bdpv_cpv,belle_beta_binned_dalitz,testbench,belle_gamma_binned_dalitz}, 
        \arabic{citeauthorvak} из которых изданы в журналах, рекомендованных ВАК\nocite{mixing,bdpv_cpv,belle_beta_binned_dalitz,testbench,belle_gamma_binned_dalitz}. 
%        \arabic{citeauthorconf} "--- в тезисах докладов\nocite{confbib1,confbib2}.
    \end{refsection}
}


 % Характеристика работы по структуре во введении и в автореферате не отличается (ГОСТ Р 7.0.11, пункты 5.3.1 и 9.2.1), потому её загружаем из одного и того же внешнего файла, предварительно задав форму выделения некоторым параметрам

\textbf{Объем и структура работы.} Диссертация состоит из~введения, четырех глав, заключения и~пяти приложений.  В первой главе рассмотрены основные феноменологические подходы к изучению нарушения \cpconj-симметрии в ускорительных экспериментах и описан экспериментальный статус изучения нарушения \cpconj-симметрии.  Во второй главе описан модельно-независимый подход к анализу многочастичных распадов $D$- и $B$-мезонов, предложена программа измерений для асимметричной $B$-фабрики, симметричной Чарм-Тау-фабрики и в эксперименте \lhcb.  В третьей главе представлено описание электрон-позитронного коллайдера \kekb и детектора \belle, описан вклад автора диссертации в подготовку электромагнитного калориметра детектора \belleii.  В четвертой главе обсуждается выполненное в эксперименте \belle первое модельно-независимое измерение параметра \pphi в распадах \bdsth, \dbkpp.  
%% на случай ошибок оставляю исходный кусок на месте, закомментированным
%Полный объём диссертации составляет  \ref*{TotPages}~страницу с~\totalfigures{}~рисунками и~\totaltables{}~таблицами. Список литературы содержит \total{citenum}~наименований.
%
Полный объём диссертации составляет \formbytotal{TotPages}{страниц}{у}{ы}{} 
с~\formbytotal{totalcount@figure}{рисунк}{ом}{ами}{ами}
и~\formbytotal{totalcount@table}{таблиц}{ей}{ами}{ами}. Список литературы содержит  
\formbytotal{citenum}{наименован}{ие}{ия}{ий}.
