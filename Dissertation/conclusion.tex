\chapter*{Заключение}						% Заголовок
\addcontentsline{toc}{chapter}{Заключение}	% Добавляем его в оглавление

%% Согласно ГОСТ Р 7.0.11-2011:
%% 5.3.3 В заключении диссертации излагают итоги выполненного исследования, рекомендации, перспективы дальнейшей разработки темы.
%% 9.2.3 В заключении автореферата диссертации излагают итоги данного исследования, рекомендации и перспективы дальнейшей разработки темы.
%% Поэтому имеет смысл сделать эту часть общей и загрузить из одного файла в автореферат и в диссертацию:

Основные результаты работы заключаются в следующем:
%% Согласно ГОСТ Р 7.0.11-2011:
%% 5.3.3 В заключении диссертации излагают итоги выполненного исследования, рекомендации, перспективы дальнейшей разработки темы.
%% 9.2.3 В заключении автореферата диссертации излагают итоги данного исследования, рекомендации и перспективы дальнейшей разработки темы.
\begin{enumerate}
  \item Изучено влияние осцилляций нейтральных $D$-мезонов на наблюдаемую величину параметра \gphi в модельно-независимом измерении в распадах \bdk, \dkpp и предложена процедура, при которой осцилляции $D$-мезонов смещают наблюдаемую величину не более, чем на~$0.2\grad$.
  \item Показано, что в предположении сохранения \cpconj-симметрии в распадах \dnkpp и при существующих экспериментальных ограничениях на величину этого нарушения, смещение наблюдаемой величины \gphi при модельно-независимом измерении в распадах \bdk, \dkpp, не превосходит $3\grad$.
  \item Показано, что модельно-независимое получение параметра \gphi в распадах \bdk, \dkpp возможно без предположения сохранения \cpconj-симметрии в распадах \dnkpp; при этом не наблюдается существенного снижения статистической чувствительности.
  \item Предложен метод модельно-независимого измерения параметров смешивания и \cpconj-нарушения в смешивании нейтральных $D$-мезонов в процессе $\ep\to \ddbar{}^{*}$ без измерения времени распада~$D$.
  \item Предложен метод модельно-независимого получения параметров смешивания и \cpconj-нарушения в смешивании нейтральных $D$-мезонов в процессе \dstpdpip, \dnkpp с измерением времени распада~$D$.
  \item Предложен метод модельно-независимого измерения \cpconj-нарушающей фазы \pphi в распадах \bdsth, \dbkpp; данный метод позволяет разрешить неопределенность, присущую измерению $2\pphi$ в переходах~\btoccs.
  \item Впервые выполнено модельно-независимое измерение фазы \pphi в распадах \bdsth, \dbkpp и получен результат $\pphi = 11.7\grad\pm7.8\grad\pm2.1\grad$, позволяющий разрешить неопределенность значения $2\pphi$ на уровне достоверности, превышающем $5$ стандартных отклонений.
  \item Подготовлен алгоритм для автоматического измерения характеристик модуля усилителя-формирователя калориметра \belleii, который был использован для проверки характеристик всех изготовленных модулей.
\end{enumerate}
% \begin{enumerate}
%   \item На основе анализа \ldots
%   \item Численные исследования показали, что \ldots
%   \item Математическое моделирование показало \ldots
%   \item Для выполнения поставленных задач был создан \ldots
% \end{enumerate}


Предложенные в данной работе модельно-независимые методы измерений могут быть использованы при выполнении прецизионных измерений в экспериментах \babar, \belle, \textrm{CLEO}, \besiii, \belleii, \lhcb, а также в экспериментах на Чарм-Тау-фабриках.  

Практическая реализуемость предложенных подходов показана в нескольких экспериментальных работах: в работе~\cite{belle_gamma_binned_dalitz} группой \belle впервые описано модельно-независимое измерение параметра~\gphi в распадах \bdk, \dkpp; группа \lhcb впервые выполнила модельно-независимое измерение параметров осцилляций $D$-мезонов в распадах \dstpdpip, \dnkpp~\cite{lhcb_dkspp_mixing}, используя предложенный в данной работе метод; в работе автора диссертации~\cite{belle_beta_binned_dalitz} описано первое модельно-независимое измерения параметра~\pphi в распадах \bdsth, \dbkpp.

Модельно-независимый анализ многочастичных распадов позволяет избавиться от трудно оцениваемой модельной неопределенности и получить вместо этого хорошо контролируемую неопределенность статистического характера, связанную с ограниченным знанием фазовых параметров \ci и \si.  Модельно-независимый подход открывает путь к выполнению прецизионных измерений с использованием многочастичных распадов в экспериментах \belleii, \lhcb и на Супер-Чарм-Тау-фабрике.

\vspace{3 cm}

В заключение я бы хотел выразить благодарность своему научному руководителю А.Е.~Бондарю за помощь в выборе задач для исследования, а также за обсуждения физических вопросов, в ходе которых сформировалось ясное стратегическое видение решения поставленных задач.  Я многому научился у своего первого научного руководителя А.О.~Полуэктова; без общения с ним выполнение представленной работы было бы невозможным.  Также хочу поблагодарить своих коллег: С.И.~Эйдельмана, А.С.~Кузьмина, П.П.~Кроковного, А.Ю.~Гармаша, А.Н.~Винокурову, В.~Савинова, M.~R\"{o}hrken и K.~Miyabayashi за ценные советы при обсуждении различных аспектов представленной работы.  Благодарю всех коллег, внесших вклад в создание и работу детектора~\belle и коллайдера~\kekb.
%  \item научному руководителю ;
%  \item первому научному руководителю А.О.~Полуэктову;
%  \item С.И.~Эйдельману, А.С.~Кузьмину, П.П.~Кроковному, А.Ю.~Гармашу, А.Н.~Винокуровой.