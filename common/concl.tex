%% Согласно ГОСТ Р 7.0.11-2011:
%% 5.3.3 В заключении диссертации излагают итоги выполненного исследования, рекомендации, перспективы дальнейшей разработки темы.
%% 9.2.3 В заключении автореферата диссертации излагают итоги данного исследования, рекомендации и перспективы дальнейшей разработки темы.
\begin{enumerate}
  \item Изучено влияние осцилляций нейтральных $D$-мезонов на наблюдаемую величину параметра \gphi в модельно-независимом измерении в распадах \bdk, \dkpp и предложена процедура, при которой осцилляции $D$-мезонов смещают наблюдаемую величину не более, чем на~$0.2\grad$.
  \item Показано, что в предположении сохранения \cpconj-симметрии в распадах \dnkpp и при существующих экспериментальных ограничениях на величину этого нарушения, смещение наблюдаемой величины \gphi при модельно-независимом измерении в распадах \bdk, \dkpp, не превосходит $3\grad$.
  \item Показано, что модельно-независимое получение параметра \gphi в распадах \bdk, \dkpp возможно без предположения сохранения \cpconj-симметрии в распадах \dnkpp; при этом не наблюдается существенного снижения статистической чувствительности.
  \item Предложен метод модельно-независимого измерения параметров смешивания и \cpconj-нарушения в смешивании нейтральных $D$-мезонов в процессе $\ep\to \ddbar{}^{*}$ без измерения времени распада~$D$.
  \item Предложен метод модельно-независимого получения параметров смешивания и \cpconj-нарушения в смешивании нейтральных $D$-мезонов в процессе \dstpdpip, \dnkpp с измерением времени распада~$D$.
  \item Предложен метод модельно-независимого измерения \cpconj-нарушающей фазы \pphi в распадах \bdsth, \dbkpp; данный метод позволяет разрешить неопределенность, присущую измерению $2\pphi$ в переходах~\btoccs.
  \item Впервые выполнено модельно-независимое измерение фазы \pphi в распадах \bdsth, \dbkpp и получен результат $\pphi = 11.7\grad\pm7.8\grad\pm2.1\grad$, позволяющий разрешить неопределенность значения $2\pphi$ на уровне достоверности, превышающем $5$ стандартных отклонений.
  \item Подготовлен алгоритм для автоматического измерения характеристик модуля усилителя-формирователя калориметра \belleii, который был использован для проверки характеристик всех изготовленных модулей.
\end{enumerate}
% \begin{enumerate}
%   \item На основе анализа \ldots
%   \item Численные исследования показали, что \ldots
%   \item Математическое моделирование показало \ldots
%   \item Для выполнения поставленных задач был создан \ldots
% \end{enumerate}
