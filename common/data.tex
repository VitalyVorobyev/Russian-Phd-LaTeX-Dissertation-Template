%%% Основные сведения %%%
\newcommand{\thesisAuthor}             % Диссертация, ФИО автора
{%
    \texorpdfstring{% \texorpdfstring takes two arguments and uses the first for (La)TeX and the second for pdf
        {Воробьев Виталий Сергеевич}% так будет отображаться на титульном листе или в тексте, где будет использоваться переменная
    }{%
        Воробьев, Виталий Сергеевич% эта запись для свойств pdf-файла. В таком виде, если pdf будет обработан программами для сбора библиографических сведений, будет правильно представлена фамилия.
    }%
}
\newcommand{\thesisUdk}                % Диссертация, УДК
{539.126.4}
\newcommand{\thesisTitle}              % Диссертация, название
{\texorpdfstring{\MakeUppercase{Модельно-независимое получение CP-нарушающих параметров с использованием когерентных состояний нейтральных D-мезонов}}{Модельно-независимое получение}}
\newcommand{\thesisTitleWithSpace}              % Диссертация, название
{\texorpdfstring{\MakeUppercase{Модельно-независимое получение CP-нарушающих параметров с использованием когерентных состояний нейтральных D-мезонов}}{Модельно-независимое получение}}
\newcommand{\thesisSpecialtyNumber}    % Диссертация, специальность, номер
{\texorpdfstring{01.04.16}{01-04-16}}
\newcommand{\thesisSpecialtyTitle}     % Диссертация, специальность, название
{\texorpdfstring{физика атомного ядра и элементарных частиц}{физика атомного ядра и элементарных частиц}}
\newcommand{\thesisDegree}             % Диссертация, научная степень
{{кандидата физико-математических наук}}
\newcommand{\thesisCity}               % Диссертация, город защиты
{Новосибирск}
\newcommand{\thesisYear}               % Диссертация, год защиты
{2016}
\newcommand{\thesisOrganization}       % Диссертация, организация
{Федеральное государственное бюджетное учреждение науки Институт ядерной физики им. Г.И.~Будкера Сибирского отделения Российской академии наук}

\newcommand{\thesisOrganizationShort}       % Диссертация, организация
{ФГБУН Институт ядерной физики им. Г.И.~Будкера СО РАН}

\newcommand{\thesisInOrganization}       % Диссертация, организация в предложном падеже: Работа выполнена в ...
{Федеральном государственном бюджетном учреждении науки Институт ядерной физики им. Г.И.~Будкера Сибирского отделения Российской академии наук}

\newcommand{\supervisorFio}            % Научный руководитель, ФИО
{Бондарь Александр Евгеньевич}
\newcommand{\supervisorRegalia}        % Научный руководитель, регалии
{доктор физико-математических наук,\\ член-корреспондент РАН, профессор}

\newcommand{\opponentOneFio}           % Оппонент 1, ФИО
{Николаенко Владимир Иванович}
\newcommand{\opponentOneRegalia}       % Оппонент 1, регалии
{кандидат физико-математических наук}
\newcommand{\opponentOneJobPlace}      % Оппонент 1, место работы
%{ФГБУ ГНЦ Институт физики высоких энергий}
{Федеральное государственное бюджетное учреждение «Государственный научный центр Российской Федерации --- Институт физики высоких энергий», г.~Протвино}
\newcommand{\opponentOneJobPost}       % Оппонент 1, должность
{ведущий научный сотрудник}

\newcommand{\opponentTwoFio}           % Оппонент 2, ФИО
{Ростовцев Андрей Африканович}
\newcommand{\opponentTwoRegalia}       % Оппонент 2, регалии
{доктор физико-математических наук}
\newcommand{\opponentTwoJobPlace}      % Оппонент 2, место работы
{Федеральное государственное бюджетное учреждение науки Институт проблем передачи информации им. А.А.~Харкевича Российской академии наук, г.~Москва}
\newcommand{\opponentTwoJobPost}       % Оппонент 2, должность
{ведущий научный сотрудник}

\newcommand{\leadingOrganizationTitle} % Ведущая организация, дополнительные строки
{Федеральное государственное бюджетное учреждение науки Физический институт им. П.Н.~Лебедева Российской академии наук, г.~Москва}

\newcommand{\defenseDate}              % Защита, дата
{<<\underline{\ 26\ }>> \underline{\smash{\ декабря\ }} 2016~г.~в~<<\underline{\ 15:45\ }>>~часов}
\newcommand{\defenseCouncilNumber}     % Защита, номер диссертационного совета
{Д 003.016.02}
\newcommand{\defenseCouncilTitle}      % Защита, учреждение диссертационного совета
{ФГБУН Института ядерной физики им. Г.И.~Будкера СО РАН}
\newcommand{\defenseCouncilAddress}    % Защита, адрес учреждение диссертационного совета
{630090, г.~Новосибирск~90, проспект Академика Лаврентьева,~11}

\newcommand{\defenseSecretaryFio}      % Секретарь диссертационного совета, ФИО
{В.С.~Фадин}
\newcommand{\defenseSecretaryRegalia}  % Секретарь диссертационного совета, регалии
{д-р~физ.-мат. наук}            % Для сокращений есть ГОСТы, например: ГОСТ Р 7.0.12-2011 + http://base.garant.ru/179724/#block_30000

\newcommand{\synopsisLibrary}          % Автореферат, название библиотеки
{ФГБУН Института ядерной физики им. Г.И.~Будкера СО РАН, г.~Новосибирск}
\newcommand{\synopsisDate}             % Автореферат, дата рассылки
{<<\underline{\ \ \ \ }>> \underline{\ \ \ \ \ \ \ \ \ \ \ \ \ } 2016~г}

\newcommand{\keywords}%                 % Ключевые слова для метаданных PDF диссертации и автореферата
{CP-симметрия, мезон, кварк, диаграмма Далица}